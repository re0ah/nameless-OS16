\documentclass[12pt]{article}
\usepackage[left=3cm, right=1cm, top=2cm, bottom=2cm]{geometry}
\usepackage[utf8]{inputenc}
\usepackage[russian]{babel}
\usepackage{caption}
\usepackage[table]{xcolor}
\usepackage{enumitem}
\usepackage{mathptmx}
\usepackage{changepage}
\usepackage{indentfirst}
\usepackage{sectsty}
\sectionfont{\centering}
\usepackage[hidelinks]{hyperref}
\hypersetup{
	colorlinks=false
}
\linespread{1.25}
\setlength{\parindent}{1.25cm}

\begin{document}
\thispagestyle{empty}
\begin{center}
	МИНИСТЕРСТВО ОБРАЗОВАНИЯ КАЛУЖСКОЙ ОБЛАСТИ
	\\ Государственное автономное профессиональное образовательное учреждение
	\\ Калужской области
	\\ <<Обнинский колледж технологий и услуг>>
	\\ КУРСОВАЯ РАБОТА
	\\ по дисциплине: "???"
	\\ на тему:
	\\ <<Исследование и анализ работы операционной системы MikeOS>>
\end{center}
\begin{flushright}
	Выполнил:
	\\ студент 3 курса группы ИС 31-18
	\\ Ермаков Р.Е.
	\\ Преподаватель:
	\\ Конобеев С.С.
\end{flushright}
\begin{center}
	г. Обнинск 2020
\end{center}

\newpage
\setcounter{page}{1}

\tableofcontents

\newpage

\section{Введение}
	MikeOS - 16-битная операционная система, написанная для процессора 80386.
MikeOS выбрана в качестве примера работы 16-битной ОС, так как она весьма
проста и имеет открытый исходный код.
\subsection{Обзор инструментов}
\subsubsection{Язык программирования}
	В качестве языка программирования выбран ассемблер, а именно его диалект
nasm. Для создания операционной системы необходимо иметь прямой доступ к
памяти и прерываниям, а поскольку нет необходимости в том, чтобы ОС была
кроссплатформенна, то ассемблер более предпочтителен чем C.
\subsubsection{Виртуальная машина}
	То, с чего будет происходить запуск операционной системы. Так как MikeOS
написана для процессора 80386, то запустить ее можно на любом компьютере
поддерживающем эту архитектуру. Существуют виртуальные машины для запуска
ОС как программы других ОС, но более предпочтительно использовать эмулятор
x86, например, QEMU.
\subsection{Обзор архитектуры процессора}
	80386 является 32-битным процессором в котором впервые появился защищенный
режим работы процессора, страничная память, аппаратная поддержка
многозадачности. Так как MikeOS, анализ которой проводится в данной работе
использует виртуальный реальный режим 8086, который впервые появляется именно
в 80386 процессоре, то и рассматриваемая архитектура - 80386. Так как MikeOS
работает в реальном режиме процессора, то именно его характеристика будет
рассмотрена далее.
\subsubsection{Набор регистров}

\subsubsection{Механизм адресации}
	
\subsection{Общее описание ОС}
\begin{enumerate}[leftmargin=2\parindent]
	\item{Архитектура ядра: \textbf{Монолитное}}
	\item{Архитектура процессора: \textbf{80386}}
	\item{Режим работы процессора: \textbf{Реальный}}
	\item{Модель памяти: \textbf{Сегментированная}}
	\item{Разрядность: \textbf{16 бит}}
	\item{Многозадачность: \textbf{Однозадачная}}
	\item{Многопользовательность: \textbf{Однопользовательская}}
	\item{Поддержка многопроцессорности: \textbf{Нет}}
	\item{UI: \textbf{CLI}}
	\item{Файловая система: \textbf{FAT12}}
\end{enumerate}

\section{Описание компонентов ОС}
\subsection{Компоненты ядра}
 	
\newpage
\section{Процесс инициализации ядра}
\subsection{Загрузка ядра в память}
	\textbf{Загрузчик операционной системы} - программа, загружаемая BIOS
чтением 512-ти байт с загрузочного сектора системного диска по адресу
0x00007C00. Необходима для первоначальной настройки (активации A20,
инициализации GDT, перехода в защищенный режим, смены видеорежима и прочего)
и загрузки операционной системы в память.
	
	Компьютер на архитектуре х86 начинает свою работу в \textbf{реальном
режиме процессора} - режиме, совместимом с старыми 16-битными процессорами
(и, соответственно, имеющий ту же, адресацию, набор регистров...). MikeOS
не выходит из этого режима и продолжает работать в нем, соответственно
нет необходимости инициализировать GDT/LDT/IDT - т.е. переходить в защищенный
режим.

\subsubsection{Структура загрузчика}
	\begin{tabular}{|l|l|}
		\hline
		Байт     & Содержание                  			  \\ \hline
		0..1	 & Инструкция перехода на код загрузчика  \\ \hline
		2..10    & Смещение, используется для названия ОС \\ \hline
		11..63   & Заголовок FAT12 (DOS 4.0 EBPB)		  \\ \hline
		64..509  & Код загрузчика              			  \\ \hline
		510..511 & Сигнатура загрузочного диска			  \\ 
		\hline
	\end{tabular}
\subsubsection{Заголовок FAT12 (DOS 4.0 EBPB)}
	EBPB (Extended BIOS Parameter Block) - расширенный блок параметров BIOS,
структура данных описывающая структуру накопителя.

	\begin{tabular}{|l|l|l|}
		\hline
		Байт     & Содержание                                         \\ \hline
		0..1     & Размер логического сектора                         \\ \hline
		2        & Количество логических секторов на кластер          \\ \hline
		3..4     & Количество зарезервированных логических секторов   \\ \hline
		5        & Количество FAT-разделов                            \\ \hline
		6..7     & Записи корневого каталога                          \\ \hline
		8..9     & Всего логических секторов                          \\ \hline
		10       & Тип устройства (media descriptor)                  \\ \hline
		11..12   & Количество логических секторов каждого FAT         \\ \hline
		13..14   & Физических секторов на дорожку                     \\ \hline
		15..16   & Количество головок                                 \\ \hline
		17..20   & Количество скрытых секторов                        \\ \hline
		21..24   & Количество больших логических секторов             \\ \hline
		25       & Физический номер устройства (0 - НГМД, 0x80 - НМД) \\ \hline
		26       & Зарезервировано                                    \\ \hline
		27       & Сигнатура расширенного раздела                     \\ \hline
		28..31   & Серийный номер тома                                \\ \hline
		32..33   & Метка тома                                         \\ \hline
		34..51   & Тип файловой системы                               \\
		\hline
	\end{tabular}

\subsubsection{Инициализация прерываний PIT (i8253, i8254)}
	Intel 8254, PIT (programmable interval timer, программируемый интервальный таймер) - таймер, имеет 3 16-битных счетчика. Модель 8254 является расширенной версией модели 8253. Используется в операционной системе nameless исключительно для того, чтобы раз в n секунд получать прерывание от таймера.

	Стандартная частота равна 1193182Hz.

	Порты:
		\begin{enumerate}[leftmargin=2\parindent]
			\item[0x40:]{канал 0}
			\item[0x41:]{канал 1}
			\item[0x42:]{канал 2}
			\item[0x43:]{командный порт}
		\end{enumerate}

	Контроллер имеет настройки, устанавливаемые с помощью следующего битового поля:
		\begin{enumerate}[leftmargin=2\parindent]
			\item[0]{BCD, десятичные цифры}
				\begin{enumerate}
					\item[0 =]{16-bit двоичный счетчик}
					\item[1 =]{16-bit двоично-десятичный счетчик}
				\end{enumerate}
			\item[1..3]{modes, режим работы}
				\begin{enumerate}
					\item[000 =]{Прерывание по переполнению счетчика}
					\item[001 =]{Programmable one-shot, Моностабильный мультивибратор}
					\item[010 =]{rate generator}
					\item[011 =]{square wave, }
					\item[100 =]{software triggered strobe}
					\item[101 =]{hardware triggered strobe (retriggerable)}
				\end{enumerate}
			\item[4..5]{read\_write, куда записывать/откуда читать и как}
				\begin{enumerate}
					\item[00 =]{counter latch commands}
					\item[01 =]{только малый знаковый байт(8..15 биты регистра)}
					\item[10 =]{только старший знаковый байт(0..7 биты регистра)}
					\item[11 =]{сначала малый знаковый байт, потом старший знаковый байт}
				\end{enumerate}
			\item[6..7]{select counter, номер регистра}
				\begin{enumerate}
					\item[00 =]{0-й счетчик}
					\item[01 =]{1-й счетчик}
					\item[10 =]{2-й счетчик}
					\item[11 =]{read-back command}
				\end{enumerate}
		\end{enumerate}
	Операционная система nameless использует такие настройки PIT: 0 011 11 00

	PIT используется в ОС nameless для переключения процессов. Каждое прерывание меняется контекст исполнения.

	Стандартная рабочая частота: 1193182Hz / 2\^12 = 291.304199Hz. Этого достаточно для создания иллюзии параллельного исполнения программ.
\subsubsection{Инициализация прерываний клавиатуры}
\newpage
\subsubsection{Сканкоды, таблица (XT scan code set)}
\hspace*{-2.8cm}
\begin{tabular}{|c|c|c|l|c|c|c|l|c|c|c|}
\hline  
	KEY & MAKE & BREAK &----& KEY  &  MAKE  & BREAK &----&    KEY  & MAKE  & BREAK \\ \hline
	 A  &  1E  &  9E   &----&  8   &   09   &   89  &----&   HOME  & E0,47 & E0,97 \\ \hline
	 B  &  30  &  B0   &----&  9   &   0A   &   8A  &----&  INSERT & E0,52 & E0,D2 \\ \hline
	 C  &  2E  &  AE   &----&  `   &   29   &   89  &----&  PG UP  & E0,49 & E0,C9 \\ \hline
	 D  &  20  &  A0   &----&  -   &   0C   &   8C  &----&  DELETE & E0,53 & E0,D3 \\ \hline
	 E  &  12  &  92   &----&  =   &   0D   &   8D  &----&  END    & E0,4F & E0,CF \\ \hline
	 F  &  21  &  A1   &----&  \   &   2B   &   AB  &----&  PG DN  & E0,51 & E0,D1 \\ \hline
	 G  &  22  &  A2   &----& BKSP &   0E   &   8E  &----& U ARROW & E0,48 & E0,C8 \\ \hline
	 H  &  23  &  A3   &----&SPACE &   39   &   B9  &----& L ARROW & E0,4B & E0,CB \\ \hline
	 I  &  17  &  97   &----&  TAB &   0F   &   8F  &----& D ARROW & E0,50 & E0,D0 \\ \hline
	 J  &  24  &  A4   &----& CAPS &   3A   &   BA  &----& R ARROW & E0,4D & E0,CD \\ \hline
	 K  &  25  &  A5   &----&L SHFT&   2A   &   AA  &----&   NUM   &   45  & C5    \\ \hline
	 L  &  26  &  A6   &----&L CTRL&   1D   &   9D  &----&   KP /  & E0,35 & E0, B5 \\ \hline
	 M  &  32  &  B2   &----&L GUI &  E0,5B & E0,DB &----&   KP *  &   37  & B7    \\ \hline
	 N  &  31  &  B1   &----&L ALT &   38   &   B8  &----&   KP -  &   4A  & CA    \\ \hline
	 O  &  18  &  98   &----&R SHFT&   36   &   B6  &----&   KP +  &   4E  & CE    \\ \hline
	 P  &  19  &  99   &----&R CTRL&  E0,1D & E0,9D &----&   KP EN & E0,1C & E0,9C \\ \hline
	 Q  &  10  &  90   &----&R GUI &  E0,5C & E0,DC &----&   KP .  &   53  & D3   \\ \hline
	 R  &  13  &  93   &----&R ALT &  E0,38 & E0,B9 &----&   KP 0  &   52  & D2   \\ \hline
	 S  &  1F  &  9F   &----& APPS &  E0,5D & E0,DD &----&   KP 1  &   4F  & CF   \\ \hline
	 T  &  14  &  94   &----& ENTER&   1C   &   9C  &----&   KP 2  &   50  & D0   \\ \hline
	 U  &  16  &  96   &----&  ESC &   01 	&   81  &----&   KP 3  &   51  & D1   \\ \hline
	 V  &  2F  &  AF   &----&  F1  &   3B 	&   BB  &----&   KP 4  &   4B  & CB   \\ \hline
	 W  &  11  &  91   &----&  F2  &   3C 	&   BC  &----&   KP 5  &   4C  & CC   \\ \hline
	 X  &  2D  &  AD   &----&  F3  &   3D 	&   BD  &----&   KP 6  &   4D  & CD   \\ \hline
	 Y  &  15  &  95   &----&  F4  &   3E 	&   BE  &----&   KP 7  &   47  & C7   \\ \hline
	 Z  &  2C  &  AC   &----&  F5  &   3F  	&   BF  &----&   KP 8  &   48  & C8   \\ \hline
	 0  &  0B  &  8B   &----&  F6  &   40  	&   C0  &----&   KP 9  &   49  & C9   \\ \hline
	 1  &  02  &  82   &----&  F7  &   41  	&   C1  &----&   ]     &   1B  & 9B   \\ \hline
	 2  &  03  &  83   &----&  F8  &   42  	&   C2  &----&   ;     &   27  & A7   \\ \hline
	 3  &  04  &  84   &----&  F9  &   43  	&   C3  &----&   '     &   28  & A8   \\ \hline     
	 4  &  05  &  85   &----&  F10 &   44  	&   C4  &----&   ,     &   33  & B3   \\ \hline   
	 5  &  06  &  86   &----&  F11 &   57   &   D7  &----&   .     &   34  & B4   \\ \hline
	 6  &  07  &  87   &----&  F12 &   58   &   D8  &----&   /     &   35  & B5   \\ \hline  
	 7  &  08  &  88   &----& 	[  &   1A   &   9A  &----&         &       &      \\
\hline
\end{tabular}
 
\begin{tabular}{|c|c|c|}
\hline
	KEY    &        MAKE       &     BREAK     \\ \hline
	PRNTSC &    E0,2A,E0,37    &  E0,B7,E0,AA  \\ \hline
	SCROLL &        46         &     C6        \\ \hline
	PAUSE  & E1,1D,45,E1,9D,C5 &    -NONE-     \\ 
\hline
\end{tabular} 

\vspace*{1 cm}
\hspace*{-1 cm}
\textbf{ACPI Scan Codes:} \\ \newline
\begin{tabular}{|c|c|c|}
\hline
	KEY   & MAKE  & BREAK \\ \hline
	Power & E0,5E & E0,DE \\ \hline
	Sleep & E0,5F & E0,DF \\ \hline
	Wake  & E0,63 & E0,E3 \\ 
\hline
\end{tabular}

\vspace*{1 cm}
\hspace*{-1 cm}
\textbf{Windows Multimedia Scan Codes:} \\ \newline
\begin{tabular}{|c|c|c|}
\hline
	KEY           & MAKE  & BREAK \\ \hline
	Next Track    & E0,19 & E0,99 \\ \hline
	Previos Track & E0,10 & E0,90 \\ \hline
	Stop          & E0,24 & E0,A4 \\ \hline
	Play/Pause    & E0,22 & E0,A2 \\ \hline
	Mute          & E0,20 & E0,A0 \\ \hline
	Volume Up     & E0,30 & E0,B0 \\ \hline
	Volume Down   & E0,2E & E0,AE \\ \hline
	Media Select  & E0,6D & E0,ED \\ \hline
	E-Mail        & E0,6C & E0,EC \\ \hline
	Calculator    & E0,21 & E0,A1 \\ \hline
	My Computer   & E0,6B & E0,EB \\ \hline
	Search        & E0,65 & E0,E5 \\ \hline
	Home          & E0,32 & E0,B2 \\ \hline
	Back          & E0,6A & E0,EA \\ \hline
	Forward       & E0,69 & E0,E9 \\ \hline
	Stop          & E0,68 & E0,E8 \\ \hline
	Refresh       & E0,67 & E0,E7 \\ \hline
	Favorites     & E0,66 & E0,E6 \\ 
\hline
\end{tabular}

\newpage
\section{Заключение}
\newpage

\section{Список используемой литературы}
\begin{enumerate}[leftmargin=2\parindent]
	\item https://github.com/mig-hub/mikeOS/blob/master/source/bootload/bootload.asm
	\item https://github.com/zavg/linux-0.01
	\item https://github.com/torvalds/linux
	\item https://github.com/gdevic/minix1
	\item https://github.com/microsoft/MS-DOS
	\item https://wiki.osdev.org
	\item Современные операционные системы, Э. Таненбаум, Х. Бос
	\item Операционные системы. Разработка и реализация. Э. Таненбаум, А. Вудхалл
	\item Intel® 64 and IA-32 Architectures Software Developer’s Manual 1, 2, 3, 4 volumes
	\item http://www.scs.stanford.edu/10wi-cs140/pintos/specs/8254.pdf
	\item https://pdos.csail.mit.edu/6.828/2014/readings/hardware/8259A.pdf
	\item https://www.tayloredge.com/reference/Interface/atkeyboard.pdf
	\item http://gvpcew.ac.in/LN-CSE-IT-22-32/ECE/3-Year/MPMC-Unit-1.pdf
	\item http://bitsavers.trailing-edge.com/components/intel/80386/231746-001\_Introduction\_to\_the\_80386\_Apr86.pdf
	\item http://bitsavers.trailing-edge.com/components/intel/80286/210498-005\_80286\_and\_80287\_Programmers\_Reference\_Manual\_1987.pdf
	\item http://read.pudn.com/downloads77/ebook/294884/FAT32%20Spec%20%28SDA%20Contribution%29.pdf
\end{enumerate}

\end{document}
